	\subsection{Types of virtual machines by \textit{Xiao-Feng}}
	
	In the book called "\textit{Advanced Design and Implementation of Virtual Machines}" by Xiao-Feng in 2016 \cite{Xiao-Feng2016}, there is no graph describing the types of classification of virtual machines, however, the existence of four types of virtual machines like this is indicated:
	
	\textbf{Type 1}: The \textbf{Full ISA virtual machine} allows full ISA level emulation or virtualization. The operating system and its applications can run on top the VM as on a real machine\cite{Xiao-Feng2016}. Example VirtualBox, QEMU, and XEN.
	
	\textbf{Type 2}: The \textbf{ABI virtual machine} allows ABI-level emulation of the processes in the guest OS. There applications can run in conjunction with native ABI applications \cite{Xiao-Feng2016}. Example: Intel's IA-32 Execution Layer on Itanium, Transmeta's Code Morphing for X86 emulation.
	
	\textbf{Type 3}: The \textbf{Virtual ISA virtual machine} provides a runtime engine so that applications encoded in the virtual ISA can run on it \cite{Xiao-Feng2016}. Example Sun Microsistem's JVM, Microsoft's Common Language Runtime, and Parrot Foundation's virtual machine \cite{Parrot}.
	
	\textbf{Type 4}: The \textbf{Language virtual machine} gives a runtime engine that runs programs written in a guest language (source). The runtime engine needs to interpret or translate the program. Example, the runtime engines for Basic, Lisp, Tcl, and Ruby.
	
	Although the study by Xiao-Feng et al. presents a classification scheme of four types, its does not indicate a hierarchical structure that clarifies how they relate, it does not have a supporting graph to facilitate understanding and, in addition, this way of classifying virtual machines does not contemplate many of the categories indicated in other taxonomies previously presented.
	
	