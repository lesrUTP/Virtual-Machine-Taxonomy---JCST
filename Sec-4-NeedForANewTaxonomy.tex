	\section {Need for a new taxonomy}\label{sec:necesidadDeUnaTaxonomia}
	
	The set of taxonomies described above have many elements that contribute to the classification of virtualization technologies. However, in each of these classification schemes aspects which need to be improved have been identified. On the other hand, each scheme offers a taxonomic approach, such as \textit{Abstraction level}, \textit{Type of VM} and \textit{Virtualization domains}. Table \ref{cuadro:resumenTrabajos} shows a summary of the classification schemes analyzed in this paper, which are identified by; author, year of publication and taxonomic approach. This table describes taxonomies published between 2005 and 2013, but does not include the period between 2013 and 2018 which needs to be updated.
	
	%JNM - are you planning to add 2013-2018? do you have a list of other taxonomies proposed in that time?  If so I think you have to cover those to have a viable paper. 
	
	\begin{table}[H]
		\centering
		\begin{tabular}{|l|c|l|}
			\hline
			\multicolumn{3}{|c|}{\textbf{Classification schemes analyzed}}\\
			\hline
			\textbf{Author} & \textbf{Year} & \textbf{Taxonomic approach} \\ 
			\hline
			Chiueh          & 2005          & Abstration level\\ 
			\hline
			Smith           & 2005          & Type of VM\\ 
			\hline
			Scope Alliance  & 2008          & Type of VM\\ 
			\hline
			Kampert         & 2010          & Domains\\ 
			\hline
			Kusnetzky       & 2011          & Domains\\ 
			\hline
			Pessolani       & 2012          & Type of VM\\ 
			\hline
			P{\'e}k         & 2013          & Type of VM\\ 
			\hline
			Ameen           & 2013          & Domains and Type of VM\\ 
			\hline
			Abdulhamid      & 2014          & Type of VM ???\\ 
			\hline
			Shuja           & 2016          & Type of VM ???\\ 
			\hline
			Xiao-Feng Li    & 2016          & Abstraction level\\ 
			\hline
			Bugnion         & 2017          & Type of VM ???\\ 
			\hline
		\end{tabular}
		\caption{Summary of classification schemes}
		\label{cuadro:resumenTrabajos}
		
	\end{table}

	The taxonomic approach \textit{Type of VM} is the most popular taxonomic approach, as demonstrated by the studies of \textit{Smith}, \textit{SCOPE Allinace}, \textit{Pessolani} and \textit{P{\'e}k}. On the other hand, \textit{Kampert} and \textit{Kusnetzky} take a different perspective, that's objective is to consider in a general way, the largest number of technological domains in which it is possible to carry out virtualization processes, hence the name \textit{Virtualization domains}. lastly, \textit{Chiueh} considers the taxonomic approach \textit{Level of abstraction} as fundamental for the categorization of virtualization technologies. When there are differences in approach to these issues, interested communities may feel confused when they are reading different authors. This is why there is a need for a taxonomy that provides a unified, organized and updated view for this topic.
	
	%JNM Refuerce esto si puede con más información sobre lo que hace su propuesta además de los objetivos. ¿Qué nuevos elementos se agregan por ejemplo?
	
	This new taxonomy should combine approaches and add new elements. This new taxonomy should be an instrument to provide support in pedagogical processes and learning in the academic community with interests in virtualization technologies. On the other hand, the new taxonomy should also facilitate the visualization of the technological ecosystem that surrounds this topic. It should help industry in its decision-making processes for the selection of these types of technologies.
	
	%JNM Creo que es una contribución reunir todas las taxonomías existentes y compararlas. ¿No tendría que argumentar que una nueva es necesaria, solo que está ubicando las existentes en un contexto para poder compararlas?
	
	%JNM Eso podría ser un argumento más fácil. ¿No es que ha habido muy pocas taxonías, pero que ha habido demasiadas y que está haciendo el trabajo de compararlas y ponerlas en contexto entre sí?
	
	
	%Creo que esta podría ser una mejor manera de posicionar las contribuciones que como una nueva taxonomía.
	
	
	% JNM tambine estan contribuyendo una visualización que los pone en contexto y un diagrama de flujo basado en esto para ayudar en la selección de tecnologías de virtualización apropiadas en un espacio confuso.
	

