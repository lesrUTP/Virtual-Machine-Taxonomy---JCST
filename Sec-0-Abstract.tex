		%Este trabajo\footnotemark[1]{} corresponde a una revisión bibliográfica sobre máquinas virtuales, buscando establecer una base taxonómica común para facilitar la comprensión de los aspectos conceptuales y funcionales de estas tecnologías; además de proveer una manera de identificar diversos tipos de máquinas virtuales existentes, también se busca que la taxonomía aquí propuesta, sea un instrumento académico para favorecer los procesos de enseñanza y aprendizaje en esta temática. Adicionalmente, se presenta un diagrama de clave taxonómica como herramienta para facilitar la elección de tecnologías de virtualización basado en la taxonomía propuesta. La revisión documental parte del cuestionamiento sobre la existencia de un modelo de clasificación o taxonomía de las tecnologías relacionadas con las máquinas virtuales.
		
		%JNM - I actually like this commented out paragraph better as a first paragraph.. I edited it a little
		This work\footnotemark[1] corresponds to a literature review on virtual machines, seeking to establish a common taxonomic base to facilitate the understanding of the conceptual and functional aspects of these technologies. In addition to providing a way to identify various types of existing virtual machines, we also intend the taxonomy proposed here to be an academic instrument to promote teaching and learning processes in this area. Additionally, a taxonomic-key diagram is shown as a tool to facilitate the choice of virtualization technologies based on the proposed taxonomy. The literature review starts with the questions about the existence of a classification model or taxonomy of the technologies related to virtual machines.
		
		%En los últimos años, varias organizaciones han utilizado las máquinas virtuales para satisfacer las necesidades tecnológicas. Sin embargo, las diferentes implementaciones de tecnologías de virtualización no tienen un enfoque común. En la literatura relacionada, se ha identificado una gran cantidad de documentación sobre procedimientos técnicos, pero muy pocas publicaciones sobre la clasificación de los tipos de virtualización. Esta situación hace que sea difícil unificar los criterios sobre la denominación y los límites conceptuales de los elementos existentes y emergentes de este conjunto de tecnologías, que confunde a los lectores y usuarios. Para lo anterior, y a partir de la revisión bibliográfica de trabajos que proponen esquemas de clasificación en la virtualización, en este trabajo \ footnotemark [1] {} propone una taxonomía de máquinas virtuales que combina los niveles de abstracción y los tipos de máquinas virtuales. Esta propuesta facilita la comprensión de los aspectos conceptuales y funcionales de estas tecnologías, y también busca ser un instrumento académico para favorecer los procesos de enseñanza y aprendizaje, y una herramienta para la industria que ayuda en la elección de tecnologías de virtualización.
		
		%JNM - This paragraph begins with a very general statement and for me that is a less powerful way to begin.. are there things in this paragraph that weren't in the other one? Maybe integrate but I think the other one is a better start
%		In recent years virtual machines have been used by various organizations to meet technological needs. 
%		However, the different implementations of virtualization technologies do not have a common approach. 
%		In the related literature, a large amount of documentation on technical procedures have been identified, but there are very few publications on the classification of virtualization types. 
%		This situation makes it difficult to unify criteria on the denomination and conceptual boundaries of existing and emerging elements of this set of technologies, which confuses readers and users. 
%		With this in mind, and starting from the bibliographic review of works that propose classification schemes on virtualization, this work \footnotemark[1]{} proposes a \textit{Virtual machine taxonomy} that combines these approaches; \textit{Abstraction level} and \textit{Types of virtual machines}. 
%		This proposal facilitates the understanding of the conceptual and functional aspects of these technologies. 
%		It also seeks to be an academic instrument to make the teaching and learning processes easier, and a tool for the industry that helps in the choice of virtualization technologies.