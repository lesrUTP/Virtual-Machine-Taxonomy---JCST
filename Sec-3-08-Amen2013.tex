	\subsection{Taxonomy of virtualization by \textit{Ameen}}
	
	\begin{figure}[!hbtp]
		\centering
		\includegraphics[width=9cm]{images/AmeenAndHamo2003.pdf}
		\vspace{-0.2cm}
		\caption{Taxonomy of virtualization by \textit{Radhwan Y. Ameen and Asmaa Y. Hamo} in 2013\footnotemark[10]{}.}
		\label{fig:TaxonomyOfVirtualizationByAmeen}
	\end{figure}
	
	\footnotetext [10] {Figure bases on the study \textit{A Survey of Server Virtualization} by Radhwan Y. Ameen and Asmaa Y. Hamo, in 2013.}

    From the perspective of the work of Ameen and Asmaa \cite{Ameen2013}, there are many different types of virtualization. In its first level is located some domains as  Mobile, Data, Memory, Desktop, Storage, Server, Network, Application, Grid and Clustering. The second level shows the types considered for the Server Virtualization as Emulation, Hosted OS, Hardware, Paravirtualization, Container and Hybrid. Finally, the third level shows the types of virtualization considered for Hardware as Type 1 and Type 2. See figure \ref{fig:TaxonomyOfVirtualizationByAmeen}.
    
    As a follow brief description about its first level components:
    
    \textbf{Mobile virtualization}: Mobile Virtualization is a thin layer of software that is embedded on a mobile device to decouple the applications and data from the underlying hardware \cite{Ameen2013}, \cite{VMware2018Website}. 
    
    \textbf{Data virtualization}: Data virtualization abstracts the source of individual data elements to allow applications to access data with a single methodology, regardless of how or where the data is stored \cite{Mann2006}.
   
    \textbf{Memory virtualization}: This consists of adding an extra level of address translation to give each VM the illusion of having zero memory address space, as provided by real hardware \cite{Ameen2013}, \cite{Waldspurger2002}.
    
    \textbf{Desktop virtualization}: It is described as the ability to display a graphical desktop from one computer system on another computer system or device \cite{Ameen2013}, \cite{VonHagen2008}.
    
    \textbf{Storage virtualization}: Is the emerging technology that creates logical abstractions of physical storage systems \cite{Ameen2013}, \cite{Bigang2005}.
    
    \textbf{Server virtualization}: It is defined as the ability to run many operating systems with isolation and independence on other operating system \cite{Ameen2013}.
    
    \textbf{Network virtualization}: It provides an abstraction layer that can decouple the physical network equipment from the delivered business services over the network \cite{Annapareddy2011}.
    
    \textbf{Application virtualization}: It allows the user to run the application using local resources without installing the application in his system completely \cite{Annapareddy2011}. Also, it provides smaller single application virtual machines that allow for emulation of a specific environment on a client system.\cite{White2010}.
    
    \textbf{Grid virtualization}: It provides a way to abstract multiple physical servers (generally heterogeneous) from the application they are running \cite{Mann2006}.
    
    \textbf{Clustering virtualization}: A cluster is a form of virtualization that makes several locally-attached physical systems appear to the application and end users as a single processing resource \cite{Ameen2013}, \cite{Mann2006}.
    
    Below is a brief description of the second level components of the Ameen and Asmaa's study. This second level focuses on the derivation of the Server category:
    
    \textbf{Emulation}: Emulation is a virtualization method in which a complete hardware architecture may be created in software. This software is able to replicate the functionality of a designated hardware processor and associated hardware systems. \cite{Mann2006}, \cite{Chiueh2005}, \cite{VonHagen2008}.

    \textbf{Hosted OS}: This is application layer virtualization, its software-only approach uses a hypervisor layer that is hosted in an underlying operating system \cite{Ameen2013}, \cite{VonHagen2008}.

    \textbf{Hardware}: This type of virtualization  is also referred to as hardware-assisted or full virtualization, the hypervisor is assisted by the processor hardware such as AMD-V or Intel VT-x processor virtualization technologies \cite{Ameen2013}, \cite{VonHagen2008}.

    \textbf{Paravirtualization}: This type of virtualization refers to a technique in which the guest OS includes modified (paravirtualized) I/O drivers for the hardware. Unlike a binary translation approach, the hypervisor does not need to trap and translate all privileged layer instructions between the guest OS and the actual server hardware. Instead, the modified guest OS makes calls directly to the virtualized I/O services and other privileged operations \cite{Ameen2013}, \cite{VonHagen2008}.

    \textbf{Container}: This type is also know as kernel-layer abstraction and refers to a techniques in witch the abstraction technology is built directly into the OS kernel rather than having a separate hypervisor layer \cite{Ameen2013} \cite{Lin2012}. 

    \textbf{Hybrid}: It is a combination of full virtualization and paravirtualization and uses input/output (I/O) acceleration techniques \cite{Ameen2013}, \cite{White2010}.

    Finally, a brief description of the third level components of the Ameen and Asmaa's study is presented. This level focuses on the derivation of the category called \textit{Hardware virtualization}:


    \textbf{Type 1}: Also known as a native or bare metal hypervisor, type 1 hypervisors run directly on the system hardware. 

    \textbf{Type 2}: Also known as a hosted hypervisor, it run on a host operating system that provides virtualization services, such as I/O device support and memory management. \cite{Ameen2013}
    
    The Ammen and Asmaa's study is closely related to the works of Kampert \cite{Kampert2010} and Kusnetzky \cite{Kusnetzky2011} and presents a classification scheme through a three-level hierarchical structure. Although this graphical representation is simple and interesting, it has many gaps because it focuses only on detailing the Server Virtualization category.



