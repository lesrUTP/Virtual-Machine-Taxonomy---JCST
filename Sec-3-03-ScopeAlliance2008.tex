	\subsection{Virtualization taxonomy by \textit{SCOPE Alliance}}
	
	\begin{figure}[H]
		\centering
		\includegraphics[width=8.5cm]{images/ScopeAlliance2008.pdf}
		\vspace{-0.2cm}
		\caption{Virtualization taxonomy by SCOPE Alliance 2008 \cite{SCOPEAlliance2008}.}
		\label{fig:TaxonomyVirtualizationSCOPEAlliance2008}
	\end{figure}
	
	The virtualization taxonomy proposed by the group \textit{SCOPE Alliance} 2008 \cite{SCOPEAlliance2008} was an extension of the research led by \textit{Smith} and \textit{Nair} in 2005 \cite{Smith2005}. It is also based on the division of virtual machines, whether they are systems or processes. Each of these categories includes conceptual elements and several examples in each of the categories, see Figure \ref{fig:TaxonomyVirtualizationSCOPEAlliance2008}. 
	
	%JNM - highlighted in what way? elaborate
	
	%LESR I changed "highlighted" by "new". This is because the work of Scope Alliance is an extension of the work of Smit and Nair. 
	
	With reference to the \textit{System VMs} (specifically with \textit{Whole system} and possibly different ISA) there are certain new elements such as, Simics \cite{Magnusson2002}, Bochs \cite{Bochs2018}, and QEMU \cite {QEMU2018}. On the other hand, when the ISA is the same, reference is made to the \textit{Hardware virtualization} that obeys the \textit{classic OS VM model}. The positioning of Type-I and Type-II hypervisors is essential to provide conceptual clarity about their functions in comparison with other taxonomies. For Type-I hypervisors, also know as \textit{Native}, it is also important to note that the taxonomy incorporates the concept of \textit{Para-virtualization} with examples, such as Xen \cite{Xen2018Website, Xen2018WebsiteCambridge}, VLX \cite{Armand2009} and the concept \textit{Full or native virtualization}. The latter can be seen either as \textit{Hardware Assisted}, such as, Xen and VLX,  or through \textit{Dynamic binary translation}, for instance, Vmware ESX. For Type-II hypervisors, the examples of \textit{VMware Workstation} \cite{VMware2018Website} and KVM \cite{KVM} are also listed. On the other hand, with reference to \textit{Process VMs} category, there are two important elements to highlight,  the \textit{Multiprogrammed System} and the \textit{Dynamic Translators}. \textit{Multiprogrammed Systems} are further classified depending on whether or not the OS provided by the underlying system is the same as the OS used by the application. If It is used the same OS, the category is called \textit{Multitask OS}, which in turn contain the category called \textit{OS Virtualization}, with examples such as \textit{Virtuozzo} \cite{OpenVZ, Virtuozzo} and \textit{Solaris Zones} \cite{SolarisZones}. If the OS is different, then the category is called \textit{Os Translator} and the taxonomy shows examples of technologies such as WABI and WINE. Returning to the \textit{Process VMs} category, when the processes are possibly based on a different ISA, the category is called \textit{Dynamic Translators}. If the VMs use the same OS the category is called  \textit{ISA \& ABI Translator} for example, \textit{FX! 32} \cite{Chernoff1998}. If the OS is different, then there are two paths, the first is called \textit{ISA \& ABI Translator} for example \textit{Transitive} \cite {Transitive} and the second is called \textit{High-level Language} for example JAVA with its JVM.
	
	Although the \textit{SCOPE Alliance} study  \cite{SCOPEAlliance2008} constitutes a significant contribution along the way to complement the taxonomy of virtualization technologies; the research does not contemplate aspects of interest such as the levels of abstraction indicated by \textit{Chiueh} \cite{ Chiueh2005} in 2005. This situation gives rise to problems of conceptual inference, in which, for example, Type-1 and Type-2 hypervisors are perceived to be at the same level of abstraction. Additionally, according to the date of publication of the study, it is necessary to carry out an extension of concepts and an update of virtualization technologies that have emerged in recent years.
	