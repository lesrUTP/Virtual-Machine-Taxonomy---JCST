	\subsection{Taxonomy of Virtualization Technologies by \textit{Chiueh}}
	\begin{figure}[H]
		\centering
		\includegraphics[width=4cm]{images/Chiueh2005.pdf}
		\vspace{-0.2cm}
		\caption{Levels of abstraction and virtualization application opportunities \cite{Chiueh2005}.}
		\label{fig:VirtualizationOpportunities}
	\end{figure}
	
	In the 2005 paper \textit{A Survey on Virtualization Technology}, Chiueh classifies virtualization technologies according to their level of abstraction \cite{Chiueh2005} including  \textit{Instruction Set Architecture Level}, \textit{Hardware Abstraction Layer}, \textit{Operating System Level}, \textit{Library Level} and \textit{Programming Language Level}. These five levels of abstraction are described below and illustrated in  Figure \ref{fig:VirtualizationOpportunities}. 
	
	\begin{itemize}
	\item\textbf{Instruction Set Architecture Level} 
	At this level of abstraction, the virtualization layer emulates an instruction set architecture (ISA) allowing virtual machines to run as if they were running directly on physical hardware. When the ISA offered by the virtualization layer differs from the ISA of the real machine, this is called emulation.   This emulation implies the interpretation of the physical instructions through software. Examples of this type of technology are Bochs \cite{Bochs2018}, QEMU \cite {QEMU2018} and BIRD \cite {Nanda2006}.
	
	\item\textbf{Hardware Abstraction Layer}
	At this level of abstraction, the virtualization layer exploits the similarity in architectures of the guest and host platforms to cut down
the interpretation latency. The virtualization does not emulate a different ISA than the underlying machine and runs directly on the underlying machine where possible.  Here, it is possible to perform independent installations of operating systems, including their applications. These applications run as if they were executed in a real environment. Examples of these technologies are VMware \cite{VMware2018Website}, Microsoft Virtual PC \cite{Honeycutt2003}, Denali \cite {Whitaker2002}, Xen \cite{Xen2018Website, Barham2003, Xen2018WebsiteCambridge}, Plex86 \cite{Plex86}, Parallel \cite{Parallels2018} and UML \cite{Dike2006, UML2006Website}. 
	
	\item \textbf{Operating System Level}
	This level is a virtualization tool that works through an operating system module to provide a virtualized system call interface  such as; Jails \cite{Biederman2006} and Ensim \cite{Ensim}.
	
	\item\textbf{Library Level}
	At this level of abstraction, the virtualization layer offers user-level libraries controlling the communication between the applications and the rest of the system. 
	That is to say, virtualization allows implementation as  an application binary interface (ABI) or an application programming interface (API). 
	For example, WINE \cite{Wine}, that allows supporting Windows applications on Unix-like systems. 
	Other examples are WABI \cite{WABI}, LXRun \cite{LXRUN} and Visual MainWin \cite{Fisher2006}.

	\item\textbf{Programming Language Level}
	This levels' virtualization technologies do not create a virtualization layer as an intermediary, but instead they implement the virtualization layer as an application that can create a virtual machine, which can be simple or complex, as is the case with the Java VM  (JVM) \cite{Lindholm1997}, Microsoft .NET common language infrastructure (CLI) \cite{Thai2003} and Parrot \cite {Parrot}. 
	
	\end{itemize}
	
	Although \textit{Chiueh} \cite{Chiueh2005} establishes an organized way to classify virtualization technologies, they do not consider the types of virtualization that exists at the same level of abstraction. In addition, it is necessary to include some technologies that have emerged in recent years.