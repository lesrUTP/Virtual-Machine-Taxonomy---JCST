	\section {Conclusions}\label{sec:conclusion}
	
	In this paper, we make the following contributions:  
	
	\begin{list}{$\bullet$}{\setlength{\leftmargin}{5pt}}
	
	    \item A review of literature about the different classification schemes for virtualization technologies proposed since 2005. These schemes have been introduced using a timeline that has allowed the identification of following taxonomic approaches \textit{Abstraction Level}, \textit{Virtual Machine Type} and \textit{Virtualization Domains}.
		
		\item When performing the analysis of each classification scheme, it was possible to identify particular weaknesses. These include; the presence of a single taxonomic approach in each scheme, and the lack of topicality considering the date of publication, and the absence in the details of the inclusion of technologies.
		
		\item The proposed taxonomy responds to the needs identified in the classification schemes analyzed. As a result, the proposal combines the \textit{Abstraction Level}, and \textit{Virtual Machine Type} approaches, giving the reader a means of visualizing the virtualization technologies relating to virtual machines. By doing so, the reader is always aware of the level of abstraction in which each technology takes place, in addition to the type of machine projected, be it a complete system or an execution environment for processes.
		
		\item The proposed taxonomy can be used in the academic field as a tool to facilitate teaching and learning processes in the academic community with interests in this type of technology, or in the business field to favour decision making when they need to implement technologies related to virtual machines. 
		
		\item Taxonomy allows the classification of virtualization technologies that are present in more than one conceptual branch, since these tools evolve meeting the needs of more than one approach by themselves, or use extensions to do so.
		
		\item Finally, a taxonomic-key diagram has been created for use by the industry as a tool to help when selecting virtualization technologies according to their different needs.

	\end{list}
